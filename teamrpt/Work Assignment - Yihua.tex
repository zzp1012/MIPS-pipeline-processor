\documentclass[a4paper]{article}
\usepackage{amsmath,amssymb,float,geometry,indentfirst,multicol,multirow,pdfpages,tabularx}
\geometry{left=3.5cm,right=3.5cm,top=3.3cm,bottom=3.3cm}
\setlength{\parindent}{2em}
\begin{document}
\section{Abstract}
In this report, we demonstrate our implementation of a 5-stage pipelined processor based on MIPS instruction set and Verilog HDL modeling.
Our project implements a pipelined processor in MIPS architecture with a forwarding unit and a hazard detecting unit to resolve hazard issues.
We use Verilog HDL to model the processor and verify our design by Vivado simulation included in this report. Besides, we also demonstrate our design on a FPGA board.
\\\\
\noindent\textbf{Keywords:} MIPS, pipelined processor, Verilog HDL
\section{Introduction}
In this project, we implement a 5-stage pipelined processor based on MIPS instruction set with a forwarding unit and a hazard detecting unit to resolve hazard issues.
MIPS is an open source instruction set architecture with high performance that is widely used in embedded systems.
Pipelined processor is implemented by a technique called instruction pipelining that is implemented by dividing instructions into a series of pipelines so that the processor's cycle time is reduced.
It is simple, reliable, and fast.

To realize instruction pipelining, we divide the handling of instructions into 5 stages: IF (instruction fetch), ID (instruction decode), EX (execute), MEM (memory access), and (register write back).
Between them, we have 4 stage registers: IF/ID register, ID/EX register, EX/MEM register, MEM/WB register.
In the first stage, we acquire PC address and instruction from instruction memory;
in the second stage, we acquire control signals, do register reading or writing if applicable, and detect hazards;
in the third stage, we do ALU (Arithmetic Logic Unit) calculation and forwarding if applicable;
in the fourth stage, we read or write data from the data memory;
in the fifth stage, we select a write-back signal.
\section{Description}
\subsection{Stage Register}
In pipelined processor, we use four stage registers to process signals: IF/ID, ID/EX, EX/MEM, MEM/WB. The names of wires and registers we use in state registers with their corresponding descriptions are list below.
\begin{table}[H]
    \centering
    \begin{tabular}{|c|l|l|}
        \hline
        &Name&Descriptions\\
        \hline
        \multirow{5}{*}{input}&clk&clock signal\\
        \cline{2-3}
        &IFID\_write&whether to write IF/ID stage register\\
        \cline{2-3}
        &IF\_flush&whether to flush IF stage\\
        \cline{2-3}
        &IF\_instr&instruction from the instruction memory in IF stage\\
        \cline{2-3}
        &IF\_pcplus4&the value of PC + 4 in IF stage\\
        \hline
        \multirow{2}{*}{output}&IFID\_instr&instruction output from IF/ID stage register\\
        \cline{2-3}
        &IFID\_pcplus4&the value of PC + 4 output from IF/ID stage register\\
        \hline
    \end{tabular}
    \caption{IF/ID stage register.}
\end{table}
\begin{table}[H]
    \centering
    \begin{tabular}{|c|l|l|}
        \hline
        &Name&Descriptions\\
        \hline
        \multirow{15}{*}{input}&clk&clock signal\\
        \cline{2-3}
        &IF\_flush&whether to flush IF stage\\
        \cline{2-3}
        &regReadData1ID&registers read data 1 in ID stage\\
        \cline{2-3}
        &regReadData2ID&registers read data 2 in ID stage\\
        \cline{2-3}
        &signExtendID&output of sign-extend in ID stage\\
        \cline{2-3}
        &registerRsID&the value of register rs in ID stage\\
        \cline{2-3}
        &registerRtID&the value of register rt in ID stage\\
        \cline{2-3}
        &registerRdID&the value of register rd in ID stage\\
        \cline{2-3}
        &aluOpID&control signal ALUOp in ID stage\\
        \cline{2-3}
        &regDstID&control signal RegDst in ID stage\\
        \cline{2-3}
        &memReadID&control signal MemRead in ID stage\\
        \cline{2-3}
        &memtoRegID&control signal MemtoReg in ID stage\\
        \cline{2-3}
        &memWriteID&control signal MemWrite in ID stage\\
        \cline{2-3}
        &aluSrcID&control signal ALUSrc in ID stage\\
        \cline{2-3}
        &regWriteID&control signal RegWrite in ID stage\\
        \hline
        \multirow{13}{*}{output}&regReadData1EX&registers read data 1 in EX stage\\
        \cline{2-3}
        &regReadData2EX&registers read data 2 in EX stage\\
        \cline{2-3}
        &signExtendEX&output of sign-extend in EX stage\\
        \cline{2-3}
        &registerRsEX&the value of register rs in EX stage\\
        \cline{2-3}
        &registerRtEX&the value of register rt in EX stage\\
        \cline{2-3}
        &registerRdEX&the value of register rd in EX stage\\
        \cline{2-3}
        &aluOpEX&control signal ALUOp in EX stage\\
        \cline{2-3}
        &regDstEX&control signal RegDst in EX stage\\
        \cline{2-3}
        &memReadEX&control signal MemRead in EX stage\\
        \cline{2-3}
        &memtoRegEX&control signal MemtoReg in EX stage\\
        \cline{2-3}
        &memWriteEX&control signal MemWrite in EX stage\\
        \cline{2-3}
        &aluSrcEX&control signal ALUSrc in EX stage\\
        \cline{2-3}
        &regWriteEX&control signal RegWrite in EX stage\\
        \hline
    \end{tabular}
    \caption{ID/EX stage register.}
\end{table}
\begin{table}[H]
    \centering
    \begin{tabular}{|c|l|l|}
        \hline
        &Name&Descriptions\\
        \hline
        \multirow{8}{*}{input}&Clock&clock signal\\
        \cline{2-3}
        &EX\_MemRead&control signal MemRead in EX stage\\
        \cline{2-3}
        &EX\_MemtoReg&control signal MemtoReg in EX stage\\
        \cline{2-3}
        &EX\_MemWrite&control signal MemWrite in EX stage\\
        \cline{2-3}
        &EX\_RegWrite&control signal RegWrite in EX stage\\
        \cline{2-3}
        &EX\_MUX8\_out&the destination register in EX stage\\
        \cline{2-3}
        &EX\_ALU\_result&ALU result as an output of ALU in EX stage\\
        \cline{2-3}
        &EX\_MUX6\_out&the value of R[rt] in EX stage\\
        \hline
        \multirow{7}{*}{output}&MEM\_MemRead&control signal MemRead in MEM stage\\
        \cline{2-3}
        &MEM\_MemtoReg&control signal MemtoReg in MEM stage\\
        \cline{2-3}
        &MEM\_MemWrite&control signal MemWrite in MEM stage\\
        \cline{2-3}
        &MEM\_RegWrite&control signal RegWrite in MEM stage\\
        \cline{2-3}
        &MEM\_MUX8\_out&the destination register in MEM stage\\
        \cline{2-3}
        &MEM\_ALU\_result&ALU result as an output of ALU in MEM stage\\
        \cline{2-3}
        &MEM\_MUX6\_out&the value of R[rt] in MEM stage\\
        \hline
    \end{tabular}
    \caption{EX/MEM stage register.}
\end{table}
\begin{table}[H]
    \centering
    \begin{tabular}{|c|l|l|}
        \hline
        &Name&Descriptions\\
        \hline
        \multirow{6}{*}{input}&Clock&clock signal\\
        \cline{2-3}
        &MEM\_RegWrite&control signal RegWrite in MEM stage\\
        \cline{2-3}
        &MEM\_MemtoReg&control signal MemtoReg in MEM stage\\
        \cline{2-3}
        &MEM\_MUX8\_out&the destination register in MEM stage\\
        \cline{2-3}
        &MEM\_Data\_memory\_Read\_data&data memory read data in MEM stage\\
        \cline{2-3}
        &MEM\_ALU\_result&ALU result as an output of ALU in MEM stage\\
        \hline
        \multirow{5}{*}{output}&WB\_MemtoReg&control signal MemtoReg in WB stage\\
        \cline{2-3}
        &WB\_RegWrite&control signal RegWrite in WB stage\\
        \cline{2-3}
        &WB\_MUX8\_out&the destination register in WB stage\\
        \cline{2-3}
        &WB\_Data\_memory\_Read\_data&data memory read data in WB stage\\
        \cline{2-3}
        &WB\_ALU\_result&ALU result as an output of ALU in WB stage\\
        \hline
    \end{tabular}
    \caption{MEM/WB stage register.}
\end{table}
\subsection{Instruction Memory \& Data Mememory Byte-accessible}
\end{document}